\section{Introduction}
\label{sec:intro}


\TheName{}~\cite{ning2019understanding} shows a severe vulnerability of the new debugging mechanism on the Arm architecture.
Although the device manufacturers can defend against the attack by hardware-based modification (e.g., implementing a hardware-assisted control of the debug registers),
they must launch a great callback of the affected devices, which triggers an unacceptable expense.
To avoid the huge cost, we should design a software-level defense for \TheName{} attack.
Then, deploying the defense can be implemented by a patch update via network, rather than the expensive callback.

To achieve the defense, we first decide where to place it.
Considering that the attacker controls the kernel (i.e., the Operating System),
we must leverage a higher privilege to monitor or prevent the \TheName{} attack. i.e., the Secure layer or the Hypervisor layer.
In this lab, we select the Hypervisor layer to deploy the defense. Specifically, we introduce an address translation, called Stage-2 translation. By configuring the translation regime, we prevent the access of the registers from the kernel-level attacker, while the access of other memory regions is unaffected.

We consider the tasks of this lab as follows:

\begin{itemize}
	\item Understand partial components of Armv8-A architecture, including the exception levels (EL) and the translation regimes.
	\item Understand how to use Raspberry PI 3 Module B+, and learn to burn the Linux kernel.
	\item Design a defense of \TheName{}, and implement it by modifying the Linux kernel. 
\end{itemize}

In this lab, you are required to submit a report including \textbf{4} 
questions. The questions will 
be raised in the following sections.